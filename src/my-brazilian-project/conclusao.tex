\section{Conclusão}

\begin{itemize}
	\item Contribuições
		\begin{itemize}
			\item Recapitular como a combinação de LLMs com recuperação (RAG) reduz alucinações e melhora a precisão factual ao consultar o DHBB.
			\item Enfatizar a importância de integrar e conectar fontes históricas para melhorar o processo de perguntas e respostas (QA).
		\end{itemize}
	\item Limitações
		\begin{itemize}
			\item Reconhecer restrições, como cobertura parcial do DHBB ou limitações de custo na hospedagem.
			\item Mencionar o risco de indexação incompleta ou recuperação ruidosa.
			\item Qualidade: realizar survey com perguntas e respostas humanas para avaliação.
		\end{itemize}
	\item Direções Futuras
		\begin{itemize}
			\item Sugerir aprimoramentos, como ampliar a cobertura para todo o DHBB, adotar modelos maiores ou mais especializados, ou incorporar algoritmos avançados de ranqueamento.
			\item Indicar a possibilidade de adicionar abordagens de grafos de conhecimento ou técnicas avançadas de desambiguação.
		\end{itemize}
\end{itemize}
