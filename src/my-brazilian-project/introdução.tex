\section{Introdução}

\subsection{Contexto e motivação}

Em 1984, o Centro de Pesquisa e Documentação de História Contemporânea do Brasil da Fundação Getulio Vargas (FGV CPDOC) publicou a primeira edição do Dicionário Histórico-Biográfico Brasileiro (DHBB). A obra se tornou a maior enciclopédia sobre a trajetória das elites políticas do país desde a década de 1930 até 1983, contendo quase 5 mil verbetes. Seu conteúdo incluía não somente os atores políticos que exerceram cargos de poder no nível federal (foco dos verbetes biográficos), mas também uma descrição de eventos históricos e instituições chave do Brasil de 1930 em diante, na forma de verbetes temáticos.

